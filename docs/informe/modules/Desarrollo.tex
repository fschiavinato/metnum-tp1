\section{Desarrollo}

En esta secci\'on expondremos el camino recorrido durante la construcci\'on del trabajo, las dificultades encontradas y la forma final del mismo.

\subsection{Implementaci\'on de algoritmos y estructuras de datos accesorias}

El primer problema a resolver fue la generaci\'on del programa que resuelva las exigencias del trabajo pr\'actico: la calibraci\'on, el c\'alculo de normales y la codificaci\'on concreta de los algoritmos te\'oricos \emph{Eliminaci\'on Gaussiana Cl\'asica} (sin pivoteo) y \emph{Factorizaci\'on de Cholesky}. 

\subsubsection {Etapa de Calibraci\'on}
Para la calibraci\'on de las fuentes lum\'inicas utilizamos las fotograf\'ias de las esferas. De esta figura conocemos su geometr\'ia, permitiendonos as\'i hallar la direcci\'on de iluminaci\'on. 

El primer paso de la calibraci\'on es calcular el radio y centro de la esfera. Utilizamos la m\'ascara provista para que solo hayan puntos negros y blancos. En esta medimos la mayor y menor componente x e y de los puntos blancos. Luego la componente x e y del centro es el promedio del m\'aximo y del m\'inimo, y el di\'ametro es la diferencia.

Una vez obtenidos los parametros de la esfera, procedemos a econtrar el punto m\'as brillante. Esta no es una tarea sencilla ya que pueden haber varios puntos brillantes (la foto puede estar saturada, o la resoluci\'on de los pixeles no permiten diferenciar) o bien la superficie de la esfera puede hacer brillar m\'as un punto alejado de la zona más brillante. Por esta raz\'on, en vez de tomar el punto m\'as brillante directamente, tomamos el promedio ponderado de todos los puntos con el brillo del p\'ixel como peso. Asi los puntos brillantes aislados influyen menos en la elecci\'on.

Con el punto m\'as brillante y con los par\'ametros de la esfera ya es posible calcular la normal. El punto es la proyecci\'on al plano de un punto de la esfera. La ecuaci\'on impl\'icita de la esfera es $F(x,y,z) = (x-x_0)^2+(y-y_0)^2+(z-z_0)^2 = r^2$ donde $(x_0,y_0,z_0)$ y $r$ son el centro y el radio de la esfera. En nuestro fijamos $z_0 = 0$(cualquier $z_0$ nos dar\'a el mismo resultado). Despejando $z = \sqrt{r^2 - (x-x0)^2 - (y-y0)^2}$. De esta forma obtenemos el punto de la esfera.

Por Teorema de la Funci\'on Impl\'icita, la normal a la superficie en el punto $(x,y,z)$ esta dado por el gradiente de $F$. En la esfera $\nabla F(x,y,z) = 2(x-x_0,y-y_0,z)$, o sea al calcular el punto m\'as brillante en la esfera ya obtuvimos la direcci\'on del haz.

\subsubsection {C\'alculo de normales por p\'ixel}

Contando con el dato de la ubicación de cada una de las doce fuentes lumínicas, se deben escoger tres fotografías de un mismo objeto, iluminadas desde tres fuentes distintas. Con esta información, se puede calcular la normal a la superficie del objeto para cada punto del objeto (representado por un píxel en cada una de las tres imágenes).

El sistema de ecuaciones formulado es el sistema (5) del enunciado del trabajo. 
Dado que las ecuaciones son las mismas para todos los píxeles de la imagen, excepto por el término independiente, la matriz que se forma puede ser factorizada con el método LU (ó PLU), de modo de ahorrar cálculos que se repiten en cada píxel.

\subsubsection{Construcci\'on del sistema de ecuaciones principal}

Una vez escogidas las im\'agenes a utilizar y calculadas las normales de cada p\'ixel, debe procederse a estimar las profundidades de cada p\'ixel. Para ello, se formula el sistema de ecuaciones presentado en la Introducci\'on, formando la matriz $M$.

La matriz M consta de una columna y dos filas (correspondientes a dos ecuaciones) por cada p\'ixel de la imagen tomada. Los p\'ixeles de una fila los ordenamos por filas de la imagen. Es decir, primero se ubican los p\'ixeles de la primer fila de la imagen, a continuaci\'on los p\'ixeles de la segunda fila de la im\'agen, etc\'etera.

Esta conformaci\'on provoca que, para los p\'ixeles de la \'ultima fila (la de m\'as abajo) de la imagen, algunos de los elementos de las ecuaciones no se encuentren definidos. Decidimos, para estas ecuaciones s\'olo incluir los elementos definidos.
Elegida esta disposici\'on de elementos y exclusi\'on de elementos no definidos, la matriz $M$ resulta una matriz esparsa banda, de dimensiones $2NxN$, donde $N$ es la cantidad de p\'ixeles de la imagen.
Por tanto, la matriz $A$ (igual a $M^t * M$) resulta de dimensiones $N X N$, tambien banda y marcadamente esparsa.

\subsubsection {Resoluci\'on del sistema de ecuaciones principal: 
Implementaciones preliminares descartadas}

El problema que se presenta, decididos los elementos que deben formar la matriz $A$, es implementar las estructuras de datos para alojarla y los algoritmos para resolver el sistema de ecuaciones.

Durante el transcurso de la resoluci\'on del trabajo, se usaron, descartaron y perfeccionaron versiones del programa, hasta llegar a su forma final.

La primer aproximaci\'on utilizada fue la m\'as natural de todas: implementar la matriz como un arreglo de arreglos de números (en lenguaje C++, esto se puede hacer con la clase vector<vector<double\textgreater\textgreater), y los algoritmos EG y CH tal como se presentan en la bibliografía citada. Esto es, recorriendo cada elemento de la matriz que corresponda al método. 
Sin embargo, dado que las imágenes provistas por la cátedra tienen aproximadamente 170.000 píxeles, una matriz $A$ consta de aproximadamente 30.000.000.000 elementos. Al intentar generar una instancia de vector<vector<double\textgreater\textgreater de 30.000.000.000 elementos en total, se saturaron las capacidades de memoria de las computadoras en que se intentó hacerlo.
Por lo tanto, esta aproximación fue descartada.

El segundo intento fue cambiar la estructura de datos de soporte de la matriz, de vector<vector<double\textgreater\textgreater, que aloja todos los valores de la matriz, a una que sólo aloje los valores no nulos, teniendo en cuenta que se trabajará con matrices esparsas. Esa estructura de datos elegida fue map<pair<int,int\textgreater ,double \textgreater, un diccionario cuyas claves representan una posición de la matriz <fila,columna\textgreater, y el valor correspondiente es el elemento de la matriz referenciado, y sólo se almacenan los valores no nulos de la matriz.
Con esta nueva estructura de datos, se logró cargar las matrices en memoria. No obstante, los algoritmos implementados seguían recorriendo todas las posiciones de la matriz, constituyendo un algoritmo de complejidad temporal cuadrático. La ejecución de estos algoritmos se proyectaba en alrededor de 40 horas en promedio.

Una tercer mejoría consistió en tratar de aprovechar la característica de ser banda de las matrices a trabajar. Se intentó hacer esto anexando a la matriz una estructura de datos en la que, para cada fila, se guardara la máxima y mínima columna de valor no nulo, y análogamente para cada columna, la máxima y mínima columna de valor no nulo. Tal estructura de datos fue vector<pair<int,int\textgreater\textgreater, que guardaba en cada posición (correspondiente a cada fila o columna), la máxima fila o columna no nula. Esto permitió alterar los algoritmos EG y CH, de modo que sólo recorran la banda de valores no nulos para cada fila o columna. De este modo, la el tiempo promedio de ejecución de los algoritmos pasó a ser de aproximadamente de 4 horas.

No obstante, se pudo realizar una mejora adicional.

\subsubsection{Resoluci\'on del sistema de ecuaciones principal: 
Implementaci\'on definitiva}

La implementación final de la resolución del sistema de ecuaciones utiliza como estructura de datos accesoria un vector que, para cada fila, indica cada una de las columnas en las que hay un valor no nulo. Análogamente, otro vector para cada columna indica cada una de las filas en las que hay un valor no nulo. Estas estructuras se implementan con dos instancias de la clase vector<set<int\textgreater\textgreater. Al ser set un conjunto ordenado, esto permite que los algoritmos EG y CH recorrer, para cada fila o columna, según corresponda, sólo las posiciones en las que hay valores no nulos.
Aprovechando el hecho de que la matriz es esparsa, y hay muy pocos elementos no nulos, esto vueve a los algoritmos lineales en cuanto a su complejidad temporal.



